% $Header: /cvsroot/latex-beamer/latex-beamer/solutions/generic-talks/generic-ornate-15min-45min.en.tex,v 1.4 2004/10/07 20:53:08 tantau Exp $

\documentclass{beamer}
\usepackage{hyperref}
% This file is a solution template for:

% - Giving a talk on some subject.
% - The talk is between 15min and 45min long.
% - Style is ornate.



% Copyright 2004 by Till Tantau <tantau@users.sourceforge.net>.
%
% In principle, this file can be redistributed and/or modified under
% the terms of the GNU Public License, version 2.
%
% However, this file is supposed to be a template to be modified
% for your own needs. For this reason, if you use this file as a
% template and not specifically distribute it as part of a another
% package/program, I grant the extra permission to freely copy and
% modify this file as you see fit and even to delete this copyright
% notice. 



\mode<presentation>
{
  \usetheme{Warsaw}
  % or ...

  \setbeamercovered{transparent}
  % or whatever (possibly just delete it)

  \setbeamertemplate{headline}{}
}



\usepackage[english]{babel}
% or whatever

\usepackage[latin1]{inputenc}
% or whatever

\usepackage{times}
\usepackage[T1]{fontenc}
% Or whatever. Note that the encoding and the font should match. If T1
% does not look nice, try deleting the line with the fontenc.


\title % (optional, use only with long paper titles)
{Autenticazione SAML su Office365}

%\subtitle
%{Presentation Subtitle} % (optional)

\author % (optional, use only with lots of authors)
{Francesco Malvezzi}
% - Use the \inst{?} command only if the authors have different
%   affiliation.

\institute[Universit� di Modena e Reggio nell'Emilia] % (optional, but mostly needed)
{
Universit� di Modena e Reggio nell'Emilia
}

%  \and
%  \inst{2}%
%  Department of Theoretical Philosophy\\
%  University of Elsewhere}
% - Use the \inst command only if there are several affiliations.
% - Keep it simple, no one is interested in your street address.

\date % (optional)
{14 maggio 2015}

%\subject{Talks}
% This is only inserted into the PDF information catalog. Can be left
% out. 



% If you have a file called "university-logo-filename.xxx", where xxx
% is a graphic format that can be processed by latex or pdflatex,
% resp., then you can add a logo as follows:

% \pgfdeclareimage[height=0.5cm]{university-logo}{university-logo-filename}
% \logo{\pgfuseimage{university-logo}}



% Delete this, if you do not want the table of contents to pop up at
% the beginning of each subsection:
%\AtBeginSubsection[]
%{
%  \begin{frame}<beamer>
%    \frametitle{Outline}
%    \tableofcontents[currentsection,currentsubsection]
%  \end{frame}
%}


% If you wish to uncover everything in a step-wise fashion, uncomment
% the following command: 

%\beamerdefaultoverlayspecification{<+->}


\begin{document}


\begin{frame}
  \titlepage
\end{frame}

%\begin{frame}
%  \frametitle{Outline}
%  \tableofcontents
  % You might wish to add the option [pausesections]
%\end{frame}


% Since this a solution template for a generic talk, very little can
% be said about how it should be structured. However, the talk length
% of between 15min and 45min and the theme suggest that you stick to
% the following rules:  

% - Exactly two or three sections (other than the summary).
% - At *most* three subsections per section.
% - Talk about 30s to 2min per frame. So there should be between about
%   15 and 30 frames, all told.

\section{Cos'� Office365}


\begin{frame}[fragile]
  \frametitle{Office365}
\begin{itemize}
\item Cloud di Microsoft;

\item rende disponibili vari servizi (mail, spazio disco, strumenti di office automation via cloud);

\item gli utenti sono in un ``active directory'' cloud che si chiama ``Azure'';
\item ha alcune funzionalit� SAML2.
\end{itemize}
\pause{}
Questo scalfisce solo la superficie.
\end{frame}

\begin{frame}[fragile]
  \frametitle{Obiettivo}

Permettere agli utenti dell'organizzazione di accedere a Office365 con le proprie credenziali:

\url{https://portal.microsoftonline.com}

\end{frame}

\section{Setup Office365}

\begin{frame}[fragile]
  \frametitle{Dimostrare la proprit� del dominio DNS}
\begin{itemize}
\item Registrare un dominio di fantasia nella gerarchia \verb+microsoftonline.com+, ad esempio: \verb+idem2015.microsoftonline.com+.

\item Associare il vostro nome dominio: Microsoft vi porr� una sfida: un record TXT da aggiungere al vostro DNS.
\end{itemize}

\end{frame}

\begin{frame}[fragile]
  \frametitle{Creare un utente sul cloud}
Anche se l'autenticazione sar� effettuata dal sistema Single Sign On locale \footnote{Microsoft ama il termine ``on premise''}, gli utenti devono esistere sul cloud, ad esempio con
\begin{itemize}
\item il tool \href{http://www.microsoft.com/en-us/server-cloud/products/forefront-identity-manager/}{Forefront Identity Manager};
\item o con chiamate powershell (ad es. \verb+New-MsolUser+).
\end{itemize}

\end{frame}


\begin{frame}[fragile]
  \frametitle{Preparare Office365 alla federazione}

Per istruire Office365 che deve fare parte di una federazione esiste un comando powershell preciso: \verb+Set-MsolDomainAuthentication+.

I suoi parametri sono:
\begin{itemize}
\item nome,
\item entityId,
\item url del servizio SAML2 POST SSO;
\item url di logout,
\item chiave pubblica (formato pem);
\end{itemize}
cio� in pratica i metadati.
\end{frame}

\section{Shibboleth-IdP}
\begin{frame}[fragile]
  \frametitle{Configurazioni di Shibboleth-IdP}

A questo punto lo Shibboleth-IdP deve essere pronto a dialogare con Office365:
\begin{itemize}
\item aggiungere una sezione apposta al \verb+relying-party.xml+;
\item scaricare i metadati da \url{https://nexus.microsoftonline-p.com/federationmetadata/saml20/federationmetadata.xml}.
\end{itemize}
\end{frame}


\begin{frame}[fragile]
Nel relying-party.xml:
\small{
\begin{verbatim}
<bean parent="RelyingPartyByName" \
  c:relyingPartyIds="urn:federation:MicrosoftOnline">
  <property name="profileConfigurations">
    <list>
      <bean parent="SAML2.SSO" \
p:encryptAssertions="false" 
p:nameIDFormatPrecedence=\
"urn:oasis:names:tc:SAML:1.1:nameid-format:unspecified" 
p:signAssertions="conditional"
p:encryptNameIDs="never" />
    </list>
  </property>
</bean>
\end{verbatim}
}
\end{frame}

\subsection{Rilascio degli atributi}
\begin{frame}[fragile]
  \frametitle{Rilascio degli attributi}

Office365 si aspetta che Shibboleth rilasci:
\begin{itemize}
\item \verb+immutableID+: in AD � il guid utente encoded base64, ma � sufficiente che sia un attributo unico (come lo uid) nel formato nameid; 
\item \verb+UserId+: equivalente allo \verb+eppn+. Personalmente l'ho ricostruito con uno scriptedAttributeDefinition per evitare di usare \verb+eppn+ che � un attributo scoped.
\end{itemize}
 
\end{frame}

\section{Grazie}
\begin{frame}[fragile]
  \frametitle{Grazie}

Grazie per l'attenzione e la pazienza.
 
\end{frame}


\end{document}


